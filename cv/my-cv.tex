\documentclass[a4paper,11pt]{article}
\usepackage[usenames,dvipsnames]{color}
\usepackage{hyperref}
\hypersetup{
    colorlinks=true,%
    citecolor=black,%
    filecolor=black,%
    linkcolor=green,%
    urlcolor=Green
}
\usepackage{amsmath}
\usepackage[lmargin=3.0cm, rmargin=1.0cm,tmargin=2.50cm,bmargin=2.50cm]{geometry}

\begin{document}
\begin{enumerate}
  \item[Personal data:] Leontiev Georgii Igorevych
  \item[Date of birth:] 29.01.1987
  \item[e-mail:]
    \href{mailto:georgii.leontiev@gmail.com}{georgii.leontiev@gmail.com}
  \item[jabber:] ($\lambda$x.folonexjabber.ru)@
  \item[skype:] folone
  \item[github:] \href{https://github.com/folone}{folone}
  \item[web:] \href{http://folone.info}{folone.info}
  \item[phone:] +380 95 169 67 82

  \item[Personal skills:]
    \begin{list}{$\bullet$}{}
      \item strive for the perfect code;
      \item willingness to know more.
    \end{list}
    \item[Position, purpose:] Scala Developer.\newline
      Willing to work on challenging projects.

    \item[Education:]
      \begin{list}{$\bullet$}{}
        \item National Taras Shevchenko University of Kyiv, 2008\newline
          Faculty of cybernetics, chair of theory and technology of programming.\newline
          Full-time, graduate (BA)\newline
          Thesis: “Semantic web”\newline
        \item National University of “Kyiv-Mohyla Academy”, 2010\newline
          Computer science faculty, chair of the software of the automated systems.\newline
          Full-time, graduate (Master)\newline
          Thesis: “Design Patterns for Object-Oriented-Functional hybrid programming languages”\newline
      \end{list}

    \item[Tech skills:]
      \begin{list}{$\bullet$}{}
        \item[Java]
          \begin{list}{$\bullet$}{}
            \item Enterprise
            \item Spring, Hibernate.
          \end{list}
        \item[Android]
          \begin{list}{$\bullet$}{}
            \item Robotium solo, frameworks, android apps.
          \end{list}
        \item[Scala]
          \begin{list}{$\bullet$}{}
            \item Core
            \item Lift Framework
            \item Akka
            \item scalaz
          \end{list}
      \end{list}

    \item[Misc, hobbies:]
      Haskell, drums, manga, jogging.
    \item[Tools, I’m used to:]
      \begin{list}{$\bullet$}{}
        \item Arch linux + xmonad;
        \item emacs (my standard setup mainly consists of: jabber.el,
          ensime, haskell-mode, nXML-mode, js2-mode, org-mode,
          \LaTeX-mode and some self-made stuff for convenience)
          \href{https://github.com/folone/.emacs}{my config on github}
        \item git.
      \end{list}

    \item[Work experience:]
      \begin{list}{$\bullet$}{}
        \item \href{http://daxx.com/en}{Daxx} (04.2011 — present)\newline
          Position: Scala developer.\newline
          Client: \href{http://www.thenewmotion.com/}{The New Motion}.\newline
          Responsibilities: I was the second person, joining the
          client's remote dev team (9 people now). Had a chance to work on/initiate/lead
          all the projects, company did.\newline
          Tech: scala, akka, scalaz, scalaxb, liftweb, dispatch,
          protobuf, mysql, cassandra, specs2, scalacheck, js,
          backbone.js, underscore.js, rabbitmq.\newline
          Projects:\newline
          \begin{list}{$\bullet$}{}
            \item Car Shering -- Enabling companies share electric
              cars among their employees.
            \item Car Online -- Putting the electric vehicle
              online. That is, get the information like accumulator
              charge state, GPS coordinates, etc. from the car to our
              servers, analyze, show that to the driver.
            \item Charge Network -- Backend to operate charge points
              across Netherlands. This is a place, where charge points
              connect to, and get managing commands (like authorizing
              users to perform a charging session based on chargepass,
              rebooting the point, etc.) and software updates from.
            \item LoveToLoad -- User portal for the chargenetwork,
              where users can get details and statistics data on their
              chargepoints and chargecards.
            \item Admin Portal -- Place, where network administrators
              manage the network through interfaces, provided by
              the chargenetwork.
            \item Winterfell -- A hub to exchange chargepoints status
              notifications across multiple networks. For example, we
              give our chargepoints statuses to oplaadpalen.nl.
            \item Bitlment -- An internal project, that processes
              chargesessions to generate billings.
            \item MSP -- Internal CRM project.
          \end{list}
        \item \href{http://www.ciklum.net/}{Ciklum} (02.2011 — 04.2011)\newline
          Position: Android Engineer.\newline
          Client: \href{http://www.mecom.com/}{Mecom}.\newline
          Responsibilities: Development of the main app backend and frontend framework.\newline
          Tech: java, android api’s.
        \item \href{http://cogniance.com/}{Cogniance Inc.} (06.2010 — 02.2011)\newline
          Positions: Android Java/Automation Test Engineer $\to$ Android Developer $\to$ Senior Software Engineer.\newline
          Clients: \href{http://www.inqmobile.com/}{INQ Mobile}, \href{http://www.laszlosystems.com/products/webtop/overview}{Laszlo Systems}, \href{http://buildabrand.com/}{Buildabrand}.\newline
          Responsibilities:\newline
          \begin{list}{$\bullet$}{}
            \item INQ Mobile: Development of the internal testing framework, unit and smoke tests for the platform in terms of that framework. Development of the backend for syncing facebook events with android calendar. Final product is INQ Cloud Touch device.\newline
            \item Laszlo Systems: Development of the mobile app for laszlo webtop server solution.\newline
            \item Buildabrand: Designing and implementing new relevant
              search architecture and algorithms.\newline
          \end{list}
          Tech: robotium solo, android platform frameworks, android apps; lucene, spring, hibernate, wordnet.
        \item \href{http://serena.com/}{Serena Software} (07.2009 — 06.2010)\newline
          Position: Software Engineer.\newline
          Project: \href{http://www.serena.com/products/sbm/}{SBM}.\newline
          Responsibilities: Development of the notification server backend, integrating new frontend in flex.\newline
          Tech: Spring, Hibernate.
        \item \href{http://cybervisiontech.com/}{CyberVision Inc.} (04.2009 — 07.2009)\newline
          Position: Junior Java Developer.\newline
          Client: \href{http://verizon.com/}{Verizon Inc.}\newline
          Responsibilities: Developing and implementing statistic tools for the system load process.\newline
          Tech: Java, ksh, PL/SQL (Oracle).
        \item \href{http://ukrcard.com.ua/}{JSC "UkrCard"} (03.2007 — 05.2008)\newline
          Position: software support department system programmer of internet-services processing centre.\newline
          Client: several Ukrainian banks.\newline
          Responsibilities: Development and support of the main processing system engine for banks-partners, creating technical documentation.\newline
          Tech: PHP, Oracle.
      \end{list}
\end{enumerate}

\end{document}
