\documentclass[12pt,a4paper,sans]{moderncv}

\moderncvstyle{classic}                             % style options are 'casual' (default), 'classic', 'oldstyle' and 'banking'
\moderncvcolor{grey}                               % color options 'blue' (default), 'orange', 'green', 'red', 'purple', 'grey' and 'black'
\usepackage[utf8]{inputenc}

\usepackage[scale=0.75]{geometry}

\name{George}{Leontiev}
\title{engineer @SoundCloud Ltd.}
\address{Berlin, Germany}
\phone[mobile]{+49~1515~234~75~86}
\homepage{folone.info}
\photo[64pt][0.4pt]{picture}
\extrainfo{%
  \includegraphics[height=12pt]{gh-mark.png}~\httplink{github.com/folone}\\%
  \includegraphics[height=12pt]{twitter.png}~\httplink[@folone]{twitter.com/folone}\\%
  \includegraphics[height=12pt]{so.png}~\httplink[folone]{stackoverflow.com/users/163423}\\%
  \emailsymbol\emaillink{($\lambda$x.folonexlambda-calcul.us)@}
}

\begin{document}

\makecvtitle

\section{Education}
\cventry{2004--2008}{Bachelor}{National Taras Shevchenko University of
  Kyiv}{Kyiv}{Faculty of cybernetics}{chair of theory and technology
  of programming}
\cventry{2008--2010}{Master}{National University of “Kyiv-Mohyla
  Academy”}{Kyiv}{Computer science faculty}{chair of the automated
  systems software development}

\section{Experience}
\subsection{Vocational}
\cventry{12.2015--present}{Engineer, Activities and Stream}{\href{http://soundcloud.com}{SoundCloud Ltd.}}{Berlin}{}{Working on scaling the highest-throughput systems in the company: systems that power user's stream and activities.}%
\cventry{02.2014--12.2015}{Engineer, Creators}{\href{http://soundcloud.com}{SoundCloud Ltd.}}{Berlin}{}{Worked on creators-focused features of the platform: services for creator stats, in-app messages, coordinating track creation, rss-feeds, transcoding flow, etc. Bootstrapped services stations and bulk content ingestion by partners}%
\cventry{01.2013--01.2014}{Quantitative Developer}{\href{http://deltamethod.com}{deltamentod GmbH}}{Berlin}{}{Working on the backend for recommendation service.\newline{}%
  Developing a recommendation system for improving online ads/ad compaigns, predicting their sucess, suggesting possible improvements. Using machine learning algorithms in scala/python. Producing functional code that scales to big amounts of data.\newline
  Tech: scala (saddle, scalaz, spire, scala-notebook, algebird, opennlp, corenlp), python (numpy, pandas, scikit-learn, pylab, ipython notebook)
Projects:\newline
\begin{list}{$\bullet$}{}
\item Westeros -- collection of algorithms that analyze text-based data for ad campaigns and make intelligent suggestions about adding keywords and negatives to campaigns/ad groups.
\item Adalyzer, Anna recommendations, BidMan and others -- collection of algorithms, working on numeric data to predict ad campaign success rates, analyze KPIs, predict structure, place bids for ad auctions, etc.
\end{list}}
\cventry{02.2011--12.2012}{Scala developer}{\href{http://thenewmotion.com}{The New Motion}}{Kyiv, Amsterdam}{}{
  I was the second person, joining the client's remote dev team (9 people now). Had a chance to work on/initiate/lead
  all the projects, company did.\newline
  Tech: scala, akka, scalaz, scalaxb, liftweb, dispatch,
  protobuf, mysql, cassandra, specs2, scalacheck, js,
  backbone.js, underscore.js, rabbitmq.\newline{}
  Projects:\newline
  \begin{list}{$\bullet$}{}
  \item Car Sharing -- Enabling companies share electric
    cars among their employees.
  \item Car Online -- Putting the electric vehicle
    online. That is, get the information like accumulator
    charge state, GPS coordinates, etc. from the car to our
    servers, analyze, show that to the driver.
  \item Charge Network -- Backend to operate charge points
    across Netherlands. This is a place, where charge points
    connect to, and get managing commands (like authorizing
    users to perform a charging session based on chargepass,
    rebooting the point, etc.) and software updates from.
  \item LoveToLoad -- User portal for the chargenetwork,
    where users can get details and statistics data on their
    chargepoints and chargecards.
  \item Admin Portal -- Place, where network administrators
    manage the network through interfaces, provided by
    the chargenetwork.
  \item Winterfell -- A hub to exchange chargepoints status
    notifications across multiple networks. For example, we
    give our chargepoints statuses to oplaadpalen.nl.
  \item Bitlment -- An internal project, that processes
    chargesessions to generate billings.
  \item MSP -- Internal CRM project.
\end{list}}
\cventry{06.2010--02.2011}{Android Java/Automation Test Engineer $\to$ Android Developer $\to$ Senior Software Engineer}{\href{http://cogniance.com/}{Cogniance inc.}}{Kyiv}{}{
  \begin{list}{$\bullet$}{}
            \item INQ Mobile: Development of the internal testing framework, unit and smoke tests for the platform in terms of that framework. Development of the backend for syncing facebook events with android calendar. Final product is INQ Cloud Touch device.\newline
            \item Laszlo Systems: Development of the mobile app for laszlo webtop server solution.\newline
            \item Buildabrand: Designing and implementing new relevant
              search architecture and algorithms.\newline
          \end{list}
          Tech: robotium solo, android platform frameworks, android apps; lucene, spring, hibernate, wordnet.}

\subsection{Opensource contributions (\href{http://osrc.dfm.io/folone}{summary})}
\cventry{}{\href{https://github.com/scala/scala}{typesafe and typelevel scala}}{\href{https://github.com/scala/scala/commits?author=folone}{List of contributions}}{}{Scala compiler and standard library}{Bufgixes and improvements. Worked on SIP-23, which is a proposal to add support for literal-based singleton types to the language. Also worked on adding the :kind feature to repl.}
\cventry{}{\href{https://github.com/twitter/util}{twitter util and finagle}}{\href{https://github.com/twitter/util/commits?author=folone}{List of contributions}}{}{Twitter libraries that we use at SoundCloud}{Worked on various parts of projects to improve and speed up mission-critical components of systems at SoundCloud.}
\cventry{}{\href{https://github.com/scalaz/scalaz}{scalaz}}{\href{https://github.com/scalaz/scalaz/commits?author=folone}{List of contributions}}{}{An extension to the core Scala library for functional programming}{Contributed in various areas, including fixing FingerTree implementation, providing examples of using Isomorphisms, Writer monad, ST}
\cventry{}{\href{https://github.com/typelevel/scalaz-contrib}{scalaz-contrib}}{}{}{Interoperability libraries and additional data structures and instances for Scalaz}{Contriburing scalaz instances for 3rd-party projects}
\cventry{}{\href{https://github.com/saddle/saddle}{saddle}}{\href{https://github.com/saddle/saddle/commits?author=folone}{List of contributions}}{}{Scala data library in spirit of python's pandas and numpy}{Some contributions to enable the hdf5 support.}
\cventry{}{\href{https://github.com/mesos/spark}{spark}}{}{}{Scala framework for iterative and interactive cluster computing}{Helped migrate the project to scala 2.10}
\cventry{}{\href{https://github.com/folone/poi.scala}{poi.scala}}{}{}{Apache poi dsl for scala}{Creator and maintainer of the project}
\cventry{}{\href{https://github.com/folone/roy-mode}{roy-mode}}{}{}{\href{http://roy.brianmckenna.org/}{Roy} mode for emacs}{Creator and maintainer of the project}
\cventry{}{\href{https://github.com/folone}{Others}}{}{}{}{I regularly contribute to other opensource projects. The latest activity is on my github profile.}

\section{Interests}
\cvitem{Math}{I am fascinated by different branches, but recently extremely interested in the category theory.}
\cvitem{Public speaking}{I'm ocasionally speaking at tech conferences. Take a look at my lanyrd profile for more details: http://lanyrd.com/profile/folone/}
\cvitem{Haskell}{Initially started to ponder around it to become a better scala developer. As a result fell in love with its ideas and semantics.}
\cvitem{Prolog}{Started to ponder around to get a hold of typelevel programming in both scala and haskell. Still pondering around.}
\cvitem{Drums, bass guitar}{Just learning}

\end{document}
