%% start of file `template.tex'.
%% Copyright 2006-2013 Xavier Danaux (xdanaux@gmail.com).
%
% This work may be distributed and/or modified under the
% conditions of the LaTeX Project Public License version 1.3c,
% available at http://www.latex-project.org/lppl/.

\documentclass[12pt,a4paper,sans]{moderncv}        % possible options include font size ('10pt', '11pt' and '12pt'), paper size ('a4paper', 'letterpaper', 'a5paper', 'legalpaper', 'executivepaper' and 'landscape') and font family ('sans' and 'roman')

% moderncv themes
\moderncvstyle{classic}                             % style options are 'casual' (default), 'classic', 'oldstyle' and 'banking'
\moderncvcolor{green}                               % color options 'blue' (default), 'orange', 'green', 'red', 'purple', 'grey' and 'black'
%\renewcommand{\familydefault}{\sfdefault}         % to set the default font; use '\sfdefault' for the default sans serif font, '\rmdefault' for the default roman one, or any tex font name
%\nopagenumbers{}                                  % uncomment to suppress automatic page numbering for CVs longer than one page

% character encoding
\usepackage[utf8]{inputenc}                       % if you are not using xelatex ou lualatex, replace by the encoding you are using

% adjust the page margins
\usepackage[scale=0.75]{geometry}
%\setlength{\hintscolumnwidth}{3cm}                % if you want to change the width of the column with the dates
%\setlength{\makecvtitlenamewidth}{10cm}           % for the 'classic' style, if you want to force the width allocated to your name and avoid line breaks. be careful though, the length is normally calculated to avoid any overlap with your personal info; use this at your own typographical risks...

% personal data
\name{George}{Leontiev}
\title{Lambda connoisseur}
\address{Berlin, Germany}
\phone[mobile]{+49~151~517~33~860}
\homepage{folone.info}
\photo[64pt][0.4pt]{picture}
\extrainfo{%
  \includegraphics[height=12pt]{gh-mark.png}~\httplink{github.com/folone}\\%
  \includegraphics[height=12pt]{twitter.png}~\httplink[@folone]{twitter.com/folone}\\%
  \includegraphics[height=12pt]{so.png}~\httplink[folone]{stackoverflow.com/users/163423}\\%
  \emailsymbol\emaillink{($\lambda$x.folonexlambda-calcul.us)@}
}

% to show numerical labels in the bibliography (default is to show no labels); only useful if you make citations in your resume
%\makeatletter
%\renewcommand*{\bibliographyitemlabel}{\@biblabel{\arabic{enumiv}}}
%\makeatother
%\renewcommand*{\bibliographyitemlabel}{[\arabic{enumiv}]}% CONSIDER REPLACING THE ABOVE BY THIS

% bibliography with mutiple entries
%\usepackage{multibib}
%\newcites{book,misc}{{Books},{Others}}
%----------------------------------------------------------------------------------
%            content
%----------------------------------------------------------------------------------
\begin{document}
%-----       resume       ---------------------------------------------------------
\makecvtitle

\section{Education}
\cventry{2004--2008}{Bachelor}{National Taras Shevchenko University of
  Kyiv}{Kyiv}{Faculty of cybernetics}{chair of theory and technology
  of programming}
\cventry{2008--2010}{Master}{National University of “Kyiv-Mohyla
  Academy”}{Kyiv}{Computer science faculty}{chair of the automated
  systems software development}

\section{Master thesis}
\cvitem{title}{\emph{Design Patterns in Object Oriented - Functional hybrid programming languages}}
\cvitem{supervisors}{Volodymyr Boublik}
\cvitem{description}{The work aims at finding and establishing well
  known/behaved abstractions known as "Design Patterns" in the world
  of hybrid OO-Functional Programming, with examples in Scala.}

\section{Experience}
\subsection{Vocational}
\cventry{01.2013--present}{Backend Developer}{\href{http://deltamethod.com}{deltamentod GmbH}}{Berlin}{}{Working on the backend for recommendation service.\newline{}%
  Developing a recommendation system for improving online ads/ad
  compaigns, using machine learning algorithms in python.\newline
  Tech: python, numpy, pandas, scikit-learn, pylab, ipython notebook}
\cventry{04.2011--12.2012}{Scala developer}{\href{http://thenewmotion.com}{The New Motion}}{Kyiv, Amsterdam}{}{
  I was the second person, joining the client's remote dev team (9 people now). Had a chance to work on/initiate/lead
  all the projects, company did.\newline
  Tech: scala, akka, scalaz, scalaxb, liftweb, dispatch,
  protobuf, mysql, cassandra, specs2, scalacheck, js,
  backbone.js, underscore.js, rabbitmq.\newline{}
  Projects:\newline
  \begin{list}{$\bullet$}{}
  \item Car Sharing -- Enabling companies share electric
    cars among their employees.
  \item Car Online -- Putting the electric vehicle
    online. That is, get the information like accumulator
    charge state, GPS coordinates, etc. from the car to our
    servers, analyze, show that to the driver.
  \item Charge Network -- Backend to operate charge points
    across Netherlands. This is a place, where charge points
    connect to, and get managing commands (like authorizing
    users to perform a charging session based on chargepass,
    rebooting the point, etc.) and software updates from.
  \item LoveToLoad -- User portal for the chargenetwork,
    where users can get details and statistics data on their
    chargepoints and chargecards.
  \item Admin Portal -- Place, where network administrators
    manage the network through interfaces, provided by
    the chargenetwork.
  \item Winterfell -- A hub to exchange chargepoints status
    notifications across multiple networks. For example, we
    give our chargepoints statuses to oplaadpalen.nl.
  \item Bitlment -- An internal project, that processes
    chargesessions to generate billings.
  \item MSP -- Internal CRM project.
\end{list}}
\cventry{02.2011--04.2011}{Android Developer}{\href{http://ciklum.net/}{Ciklum}}{Kyiv}{}{Development of the main app backend and frontend framework.\newline
  Tech: java, android api’s.}
\cventry{06.2010--02.2011}{Android Java/Automation Test Engineer $\to$ Android Developer $\to$ Senior Software Engineer}{\href{http://cogniance.com/}{Cogniance inc.}}{Kyiv}{}{
  \begin{list}{$\bullet$}{}
            \item INQ Mobile: Development of the internal testing framework, unit and smoke tests for the platform in terms of that framework. Development of the backend for syncing facebook events with android calendar. Final product is INQ Cloud Touch device.\newline
            \item Laszlo Systems: Development of the mobile app for laszlo webtop server solution.\newline
            \item Buildabrand: Designing and implementing new relevant
              search architecture and algorithms.\newline
          \end{list}
          Tech: robotium solo, android platform frameworks, android apps; lucene, spring, hibernate, wordnet.}

\subsection{Opensource contributions}
\cventry{}{\href{https://github.com/scalaz/scalaz}{scalaz}}{\href{https://github.com/scalaz/scalaz/commits?author=folone}{List
    of contributions}}{}{An extension to the core Scala library for
  functional programming}{Contributed in various areas, including fixing FingerTree implementation,
providing examples of using Isomorphisms, Writer monad, ST}
\cventry{}{\href{https://github.com/typelevel/scalaz-contrib}{scalaz-contrib}}{}{}{Interoperability libraries and additional data structures and instances for Scalaz}{Contriburing scalaz instances for 3rd-party projects}
\cventry{}{\href{https://github.com/mesos/spark}{spark}}{}{}{Scala framework for iterative and interactive cluster computing}{Helped migrate the project to scala 2.10}
\cventry{}{\href{https://github.com/folone/poi.scala}{poi.scala}}{}{}{Apache poi dsl for scala}{Creator and maintainer of the project}
\cventry{}{\href{https://github.com/folone/roy-mode}{roy-mode}}{}{}{\href{http://roy.brianmckenna.org/}{Roy} mode for emacs}{Creator and maintainer of the project}
\cventry{}{\href{https://github.com/folone/typelevel-activator}{typelevel-activator}}{}{}{A project to get started and get comfortable with \href{http://typelevel.org/}{typelevel.scala} stack}{Creator and maintainer of the project}

\section{Tech skills}
\cvitemwithcomment{}{Scala}{
  \begin{list}{$\bullet$}{}
  \item Core
  \item Lift Framework
  \item Akka (1.x, 2.0.x, 2.1)
  \item scalaz (6.x, 7.0.0)
  \item specs2, scalacheck
  \item sbt, maven
  \item Play2! framework
  \item spark
  \item spray
  \end{list}}
\cvitemwithcomment{}{JS}{
  \begin{list}{$\bullet$}{}
  \item jquery
  \item backbone.js
  \item roy, contributed to the project's ecosystem (maven plugin, emacs mode)
  \item have interest in ray, as well as idris and agda js backends
  \end{list}}
\cvitemwithcomment{}{Java}{
  \begin{list}{$\bullet$}{}
  \item Enterprise
  \item Spring, Hibernate
  \item java is here only because I have some experience in it, I
    don't really want a job, where I'd have to code in it.
  \end{list}}

\section{Tools}
\cvdoubleitem{system}{Arch linux + xmonad}{}{}
\cvdoubleitem{coding}{emacs} {setup}{mainly consists of: evil-mode, scala2-mode
  ensime (with scala-mode), haskell-mode, nXML-mode, js2-mode,
  python-mode (with pep8 and pylint), org-mode, \LaTeX-mode, jabber.el and some self-made stuff for convenience}
\cvdoubleitem{version controll}{git, mercurial}{}{}

\section{Talks}
\cvitemwithcomment{\href{https://speakerdeck.com/folone/scalaz-talk}{Scalaz:
Learn You Yet Another Real World Gentle Haskell}}{Scala User Group
  Berlin-Brandenburg}{An introductory talk about the high-level
  abstractions provided by scalaz. Demo: the IO monad.}
\cvitemwithcomment{\href{https://speakerdeck.com/folone/scala-solution-for-the-funclub-berlin-meeting}{funclub
meetup solution in scala}}{Funclub Berlin}{A solution for the programming assignment,
  done in scala, with effect-controlling via scalaz}
\cvitemwithcomment{\href{http://www.folone.info/docs/scalaz-presentation.html}{Scalaz 6.0.4 talk} (in Russian)}{Scala User Group Ukraine}{An
  introductory talk about utilising abstractions from scalaz, version 6.0.4}

\section{Interests}
\cvitem{Math}{I am fascinated by different branches, but recently
  extremely interested in the category theory.}
\cvitem{Haskell}{Initially started to ponder around it to become a
  better scala developer. As a result fell in love with its ideas and semantics.}
\cvitem{Drums, bass guitar}{Just learning}

\section{Things I do every day}
\cvlistitem{state and prove theorems of formal constructivist logic (\href{http://en.wikibooks.org/wiki/Haskell/The_Curry-Howard_isomorphism}{proof})}
\cvlistitem{learn cool new things}

%\section{References}
%\begin{cvcolumns}
%  \cvcolumn{Category 1}{\begin{itemize}\item Person 1\item Person 2\item Person 3\end{itemize}}
%  \cvcolumn{Category 2}{Amongst others:\begin{itemize}\item Person 1, and\item Person 2\end{itemize}(more upon request)}
%  \cvcolumn[0.5]{All the rest \& some more}{\textit{That} person, and \textbf{those} also (all available upon request).}
%\end{cvcolumns}

% Publications from a BibTeX file without multibib
%  for numerical labels: \renewcommand{\bibliographyitemlabel}{\@biblabel{\arabic{enumiv}}}% CONSIDER MERGING WITH PREAMBLE PART
%  to redefine the heading string ("Publications"): \renewcommand{\refname}{Articles}
%\nocite{*}
%\bibliographystyle{plain}
%\bibliography{publications}                        % 'publications' is the name of a BibTeX file

% Publications from a BibTeX file using the multibib package
%\section{Publications}
%\nocitebook{book1,book2}
%\bibliographystylebook{plain}
%\bibliographybook{publications}                   % 'publications' is the name of a BibTeX file
%\nocitemisc{misc1,misc2,misc3}
%\bibliographystylemisc{plain}
%\bibliographymisc{publications}                   % 'publications' is the name of a BibTeX file

%\clearpage
%-----       letter       ---------------------------------------------------------
% recipient data
%\recipient{Company Recruitment team}{Company, Inc.\\123 somestreet\\some city}
%\date{January 01, 1984}
%\opening{Dear Sir or Madam,}
%\closing{Yours faithfully,}
%\enclosure[Attached]{curriculum vit\ae{}}          % use an optional argument to use a string other than "Enclosure", or redefine \enclname
%\makelettertitle

%Lorem ipsum dolor sit amet, consectetur adipiscing elit. Duis ullamcorper neque sit amet lectus facilisis sed luctus nisl iaculis. Vivamus at neque arcu, sed tempor quam. Curabitur pharetra tincidunt tincidunt. Morbi volutpat feugiat mauris, quis tempor neque vehicula volutpat. Duis tristique justo vel massa fermentum accumsan. Mauris ante elit, feugiat vestibulum tempor eget, eleifend ac ipsum. Donec scelerisque lobortis ipsum eu vestibulum. Pellentesque vel massa at felis accumsan rhoncus.

%Suspendisse commodo, massa eu congue tincidunt, elit mauris pellentesque orci, cursus tempor odio nisl euismod augue. Aliquam adipiscing nibh ut odio sodales et pulvinar tortor laoreet. Mauris a accumsan ligula. Class aptent taciti sociosqu ad litora torquent per conubia nostra, per inceptos himenaeos. Suspendisse vulputate sem vehicula ipsum varius nec tempus dui dapibus. Phasellus et est urna, ut auctor erat. Sed tincidunt odio id odio aliquam mattis. Donec sapien nulla, feugiat eget adipiscing sit amet, lacinia ut dolor. Phasellus tincidunt, leo a fringilla consectetur, felis diam aliquam urna, vitae aliquet lectus orci nec velit. Vivamus dapibus varius blandit.

%Duis sit amet magna ante, at sodales diam. Aenean consectetur porta risus et sagittis. Ut interdum, enim varius pellentesque tincidunt, magna libero sodales tortor, ut fermentum nunc metus a ante. Vivamus odio leo, tincidunt eu luctus ut, sollicitudin sit amet metus. Nunc sed orci lectus. Ut sodales magna sed velit volutpat sit amet pulvinar diam venenatis.

%Albert Einstein discovered that $e=mc^2$ in 1905.

%\[ e=\lim_{n \to \infty} \left(1+\frac{1}{n}\right)^n \]

%\makeletterclosing

%\clearpage\end{CJK*}                              % if you are typesetting your resume in Chinese using CJK; the \clearpage is required for fancyhdr to work correctly with CJK, though it kills the page numbering by making \lastpage undefined
\end{document}

%% end of file `template.tex'.
